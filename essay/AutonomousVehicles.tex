%
% File coling2014.tex
%
% Contact: jwagner@computing.dcu.ie
%%
%% Based on the style files for ACL-2014, which were, in turn,
%% Based on the style files for ACL-2013, which were, in turn,
%% Based on the style files for ACL-2012, which were, in turn,
%% based on the style files for ACL-2011, which were, in turn, 
%% based on the style files for ACL-2010, which were, in turn, 
%% based on the style files for ACL-IJCNLP-2009, which were, in turn,
%% based on the style files for EACL-2009 and IJCNLP-2008...

%% Based on the style files for EACL 2006 by 
%%e.agirre@ehu.es or Sergi.Balari@uab.es
%% and that of ACL 08 by Joakim Nivre and Noah Smith

\documentclass[11pt]{article}
\usepackage{coling2014}
\usepackage{times}
\usepackage{url}
\usepackage{latexsym}
\usepackage{hyperref}

%\setlength\titlebox{5cm}

% You can expand the titlebox if you need extra space
% to show all the authors. Please do not make the titlebox
% smaller than 5cm (the original size); we will check this
% in the camera-ready version and ask you to change it back.


\title{Title}

\author{First Author \\
  Affiliation / Address line 1 \\
  Affiliation / Address line 2 \\
  Affiliation / Address line 3 \\
  {\tt email@domain} \\\And
  Second Author \\
  Affiliation / Address line 1 \\
  Affiliation / Address line 2 \\
  Affiliation / Address line 3 \\
  {\tt email@domain} \\}

\date{}

\begin{document}
\maketitle
\begin{abstract}
Abstract
\end{abstract}

\section{Introduction}
\subsection{Positive things about AVs}
...
\subsection{History}
...

\section{How AVs work}
\subsection{Sensors}
different types, how they work, Camera 

\subsection{Software}
Big data, What they look for, Testing

\section{Social and Ethical Implications of AVs}

\subsection{Everyday Life and Social Implications}
\cite{Rand:16}
\begin{itemize}
\item most likely \textbf{changed patterns of land use}: low-density patterns of land-use surrounding metropolitan regions due to facilitated mobility.  
\item \textbf{Mobility for those unable or unwilling to drive}: independence, reduction of social isolation, ... possibly connect this to the gap between rich and poor.
\item \textbf{energy and emissions}: the overall effect is unclear, total VMT will likely increase, but the technology offers the possibility to decrease emissions as well. 
\end{itemize}

\subsection{Economics}
\begin{itemize}
\item \textbf{Gap between rich and poor}: AVs will likely be very expensive until they have become ``mainstream'', so this possibly safe, fast and reliable and convenient mean of transportation might be only available to the richer parts of the population, increasing said gap. Compare \cite[p. 39]{Rand:16}
\item \textbf{Lost jobs}: A variety of professions related to the classical automobile industry will likely be affected by the introduction of AVs: People working in transportation and logistics, insurances, ... Since AVs could facilitate the introduction of alternative fuels (or this might happen simultaneously anyway), jobs related to combustion enginges might be lost as well. My personal prediction: More jobs will be rendered obsolete by AVs than new jobs will be created. Possible policy reactions might be discussed here, even radical ones like an unconditional income. Compare \cite[p. 40ff]{Rand:16}
\end{itemize}
\subsection{Legal}
\begin{itemize}
\item \textbf{Liability implications}: The manufacturer's liability will likely increase (danger (discussion?) of slowdown of technology adoption). This might lead to \textbf{vehicles as a service}. \textbf{Closer monitoring of driver behaviour} is mentioned in the RAND report as well, this might be a point of discussion.
\item \textbf{Policy toolkit}: federal insurance backstop, or simply assign liability to human drivers? \textit{tort preemption??}
\end{itemize}

\subsection{Ethical Issues}
...

\section{Conclusion}
\subsection{Future of AVs}
...
\subsection{Our Predictions}
...


%\bibliographystyle{acl}
%\bibliography{acl2014}

\begin{thebibliography}{}

\bibitem[\protect\citename{Anderson \bgroup et al.\egroup }2016]{Rand:16}
Anderson, James M., Nidhi Kalra, Karlyn D. Stanley, Paul Sorensen, Constantine Samaras and Oluwatobi A. Oluwatola.
\newblock 2016.
\newblock {\em Autonomous Vehicle Technology: A Guide for Policymakers.}.
\newblock Prentice-{Hall}, Englewood Cliffs, NJ.
\newblock \href{http://www.rand.org/pubs/research\_reports/RR443-2.html}{http://www.rand.org/pubs/research\_reports/RR443-2.html}


% ---------------------------------------------------------------
% The following citations are just examples and ARE TO BE REMOVED
% ---------------------------------------------------------------

\bibitem[\protect\citename{Aho and Ullman}1972]{Aho:72}
Alfred~V. Aho and Jeffrey~D. Ullman.
\newblock 1972.
\newblock {\em The Theory of Parsing, Translation and Compiling}, volume~1.
\newblock Prentice-{Hall}, Englewood Cliffs, NJ.

\bibitem[\protect\citename{{American Psychological Association}}1983]{APA:83}
{American Psychological Association}.
\newblock 1983.
\newblock {\em Publications Manual}.
\newblock American Psychological Association, Washington, DC.

\bibitem[\protect\citename{{Association for Computing Machinery}}1983]{ACM:83}
{Association for Computing Machinery}.
\newblock 1983.
\newblock {\em Computing Reviews}, 24(11):503--512.

\bibitem[\protect\citename{Chandra \bgroup et al.\egroup }1981]{Chandra:81}
Ashok~K. Chandra, Dexter~C. Kozen, and Larry~J. Stockmeyer.
\newblock 1981.
\newblock Alternation.
\newblock {\em Journal of the Association for Computing Machinery},
  28(1):114--133.

\bibitem[\protect\citename{Gusfield}1997]{Gusfield:97}
Dan Gusfield.
\newblock 1997.
\newblock {\em Algorithms on Strings, Trees and Sequences}.
\newblock Cambridge University Press, Cambridge, UK.

\end{thebibliography}

\end{document}

%%% Local Variables:
%%% mode: latex
%%% TeX-master: t
%%% End:
