%
% File coling2014.tex
%
% Contact: jwagner@computing.dcu.ie
%%
%% Based on the style files for ACL-2014, which were, in turn,
%% Based on the style files for ACL-2013, which were, in turn,
%% Based on the style files for ACL-2012, which were, in turn,
%% based on the style files for ACL-2011, which were, in turn, 
%% based on the style files for ACL-2010, which were, in turn, 
%% based on the style files for ACL-IJCNLP-2009, which were, in turn,
%% based on the style files for EACL-2009 and IJCNLP-2008...

%% Based on the style files for EACL 2006 by 
%%e.agirre@ehu.es or Sergi.Balari@uab.es
%% and that of ACL 08 by Joakim Nivre and Noah Smith

\documentclass[11pt]{article}
\usepackage{coling2014}
\usepackage{times}
\usepackage{url}
\usepackage{latexsym}
\usepackage{hyperref}
\usepackage{todonotes}

%\setlength\titlebox{5cm}

% You can expand the titlebox if you need extra space
% to show all the authors. Please do not make the titlebox
% smaller than 5cm (the original size); we will check this
% in the camera-ready version and ask you to change it back.

\newcommand{\TODO}[1]{\todo[inline]{{\footnotesize #1}}}

\title{Title}

\author{First Author \\
  Affiliation / Address line 1 \\
  Affiliation / Address line 2 \\
  Affiliation / Address line 3 \\
  {\tt email@domain} \\\And
  Second Author \\
  Affiliation / Address line 1 \\
  Affiliation / Address line 2 \\
  Affiliation / Address line 3 \\
  {\tt email@domain} \\}

\date{}

\begin{document}
\maketitle
\begin{abstract}
Abstract
\end{abstract}

\TODO{add todos and comments wherever you like, like this!}

\section{Introduction}
\subsection{Positive things about AVs}
\TODO{Joel: Make sure to not have ovelap, but instead be an introduction to part 3.1}
Safety \\
Fuel consumption and environment \\
Congestion \\
Incapability of driving \\
Free time \\
Design-possibilities 

\cite{Rand:16}

\subsection{History}
Radio-controlled cars ~1920\\
\cite{The Milwaukee Sentinel:26}\\
Curciuts, wires and devices imbedded in the roadway ~1930-1960\\
\cite{Press-Courier:60}\\
Neural Networks ~1990\\
95-98\% autonomous ~1990\\
DARPA / DARPA Grand Challenge ~2000\\
\cite{Thrun:06}\\
Autonomous cars on the roads ~2010\\
\cite{eWeek:14}

\section{How AVs work}
\subsection{Sensors}
different types, how they work, Camera
\begin{itemize}
	\item Extereoceptive
	\item Proprioceptive
\end{itemize}
\subsection{Software}
Cars create very specific maps
    maps take into account everything around them
    cars send back data they collect to update location map information
Cars figure out where they are by finding where they fit on the map, much like if you had a section of a map you could fit it in the context of the entire map
software maps all things around the car
looks at cyclysts, pedestrians, other cars, solid objects
Also trafic cones, people in vests, etc. 
Anticipates what each thing is going to do and when. 
Readjusts bases on what actually happens 
If it is not sure of something it slows to a stop


\section{Social and Ethical Implications of AVs}
\TODO{Gerd: I'm pretty sure this is more than enough material for this section already. If we include all the bullet points I listed here, I will not be able to go into any detail (which is what we want?)}.

\subsection{Everyday Life and Social Implications}
\cite{Rand:16}
\begin{itemize}
\item most likely \textbf{changed patterns of land use}: low-density patterns of land-use surrounding metropolitan regions due to facilitated mobility.
\item \textbf{Mobility for those unable or unwilling to drive}: independence, reduction of social isolation, ... possibly connect this to the gap between rich and poor.
\item \textbf{Free time}: Time spent in an AV could be used as work time or free time to a certain extent.
\item \textbf{Energy and emissions}: the overall effect is unclear, total VMT will likely increase, but the technology offers the possibility to decrease emissions as well. 
\end{itemize}

\subsection{Economics}
\begin{itemize}
\item \textbf{Lost jobs}: A variety of professions related to the classical automobile industry will likely be affected by the introduction of AVs: People working in transportation and logistics, insurances, ... Since AVs could facilitate the introduction of alternative fuels (or this might happen simultaneously anyway), jobs related to combustion enginges might be lost as well. My personal prediction: More jobs will be rendered obsolete by AVs than new jobs will be created. Possible policy reactions might be discussed here, even an unconditional income. Compare \cite[p. 40ff]{Rand:16}
\item \textbf{Gap between rich and poor}: AVs will likely be very expensive until they have become ``mainstream'', so this possibly safe, fast and reliable and convenient mean of transportation might be only available to the richer parts of the population, increasing said gap. Compare \cite[p. 39]{Rand:16}
\end{itemize}

\subsection{Legal}
\begin{itemize}
\item \textbf{Liability implications}: The manufacturer's liability will likely increase (danger of slowdown of technology adoption - possible discussion: how big a danger is this?). This might lead to \textbf{vehicles as a service}. \textbf{Closer monitoring of driver behaviour} is mentioned in the RAND report as well, this might be a point of discussion.
\item \textbf{Policy toolkit}: federal insurance backstop, or simply assign liability to human drivers? \textit{tort preemption??}
\end{itemize}

\subsection{Ethical Issues}
\begin{itemize}
\item Automated vehicles would almost certainly crash
\item An automated vehicle's decisions that preceded certain crashes has a moral component. 
\item There is no obvious way to encode complex human morals effectively in software. 
\end{itemize}
\cite{Goodall:14}

\subsubsection{Possible AV ethics approaches}

\begin{itemize}
\item \textbf{Rational Approaches}. Deontology (AV must adhere to a set of rules), consequentialism (system's goal is to maximize some benefit). \textbf{Problems}: Incompleteness of any set of rules and the difficulty involved in the articulation of complex human ethics as a set of rules.
\item \textbf{AI Approaches}. Idea: Train a neural network on a combination of simulations and recordings of real crashes, with human feedback on the ethical response. \textbf{Problems}: Inconsistency between how humans actually behave and what they believe, and lacking traceability (why did the NN decide the way it did?? Risk of missing transparency and manipulation).
\end{itemize}

\subsubsection{Proposed Ethical Vehicle Deployment Strategy}
Analogy: moral education of a child.
\begin{enumerate}
\item Phase 1: Rational Ethics
\item Phase 2: Hybrid rational and AI approach (rule system from Phase 1 should remain in place as boundary)
\item Phase 3: Feedback with Natural Language (active field of research)
\end{enumerate}

\section{Conclusion}
\subsection{Future of AVs}
...
\subsection{Our Predictions}
...


%\bibliographystyle{acl}
%\bibliography{acl2014}

\begin{thebibliography}{}

\bibitem[\protect\citename{Anderson \bgroup et al.\egroup }2016]{Rand:16}
Anderson, James M., Nidhi Kalra, Karlyn D. Stanley, Paul Sorensen, Constantine Samaras and Oluwatobi A. Oluwatola.
\newblock 2016.
\newblock {\em Autonomous Vehicle Technology: A Guide for Policymakers.}.
\newblock Prentice-{Hall}, Englewood Cliffs, NJ.
\newblock \href{http://www.rand.org/pubs/research\_reports/RR443-2.html}{http://www.rand.org/pubs/research\_reports/RR443-2.html}

\bibitem[\protect\citename{Goodall}2014]{Goodall:14}
Goodall, Noah.
\newblock 2014.
\newblock {\em Ethical decision making during automated vehicle crashes}, volume~1. 
\newblock Transportation Research Record: Journal of the Transportation Research Board 2424: 58-65.

\bibitem[\protect\citename{Maisto, Michelle}2016]{eWeek:14}
Michelle Maistro.
\newblock 2014-01-06.
\newblock {\em Induct Now Selling Navia, First Self-Driving Commercial Vehicle}.
\newblock eWeek.
\newblock \href{http://www.eweek.com/innovation/induct-now-selling-navia-first-self-driving-commercial-vehicle.html}{http://www.eweek.com/innovation/induct-now-selling-navia-first-self-driving-commercial-vehicle.html}

\bibitem[\protect\citename{The Milwaukee Sentinel}1926]{The Milwaukee Sentinel:26}
The Milwaukee Sentinel.
\newblock 1926-12-08.
\newblock {\em 'Phantom Auto' Will Tour City}.
\newblock Google News Archive.
% TODO: Supress error \newblock \href{https://news.google.com/newspapers?id=unBQAAAAIBAJ&sjid=QQ8EAAAAIBAJ&pg=7304,3766749}{https://news.google.com/newspapers?id=unBQAAAAIBAJ&sjid=QQ8EAAAAIBAJ&pg=7304,3766749}

\bibitem[\protect\citename{S. Thrun \bgroup et al. \egroup}2006]{Thrun:06}
Sebastian Thrun, Mike Montemerlo, Hendrik Dahlkamp, David Stavens, Andrei Aron, James Diebel, Philip Fong, John Gale, Morgan Halpenny, Gabriel Hoffmann, Kenny Lau, Celia Oakley, Mark Palatucci, Vaughan Pratt, Pascal Stang, Sven Strohband, Cedric Dupont, Lars-Erik Jendrossek, Christian Koelen, Charles Markey, Carlo Rummel, Joe van Niekerk, Eric Jensen, Philippe Alessandrini, Gary Bradski, Bob Davies, Scott Ettinger, Adrian Kaehler, Ara Nefian and Pamela Mahoney.
\newblock 2006.
\newblock {\em Stanley: The robot that won the DARPA Grand Challenge}.
\newblock Journal of Field Robotics, volume 23, issue 9.

\bibitem[\protect\citename{The Press-Courier}1960]{Press-Courier:60}
Doc Quigg
\newblock 1960-06-07.
\newblock {\em Reporter Rides Driverless Car}.
\newblock Google News Archive.
% TODO: Supress error \newblock \href{https://news.google.com/newspapers?id=vUpeAAAAIBAJ&sjid=3WANAAAAIBAJ&pg=6885,3667738&hl=en}{https://news.google.com/newspapers?id=vUpeAAAAIBAJ&sjid=3WANAAAAIBAJ&pg=6885,3667738&hl=en}

\bibitem[\protect\citename{B. Schweber}]{Mouser}
Bill Schweber.
\newblock {\em The Autonomous Car: A Diverse Array of Sensors Drives Navigation, Driving, and Performance}.
\newblock Mouser Electronics.
\newblock \href{http://eu.mouser.com/applications/autonomous-car-sensors-drive-performance/}{http://eu.mouser.com/applications/autonomous-car-sensors-drive-performance/}

\bibitem[\protect\citename{T. Hellstr\"{o}m} 2009]{Umea:09}
Thomas Hellstr\"{o}m
\newblock 2009.
\newblock {\em Sensors for autonomous vehicles}.
\newblock Ume{\aa} University.
\newblock
\href{http://www8.cs.umu.se/kurser/5DV029/HT09/handouts/Sensors for autonomous vehicles .pdf}{http://www8.cs.umu.se/kurser/5DV029/HT09/handouts/Sensors for autonomous vehicles .pdf}

\bibitem[\protect\citename{EIJ} 2012]{EIJ:12}
Earth Imaging Journal
\newblock 2012-01-24.
\newblock {\em LiDAR Boosts Brain Power for Self-Driving Cars}.
\newblock Earth Imaging Journal.
\newblock
\href{http://eijournal.com/resources/lidar-solutions-showcase/lidar-boosts-brain-power-for-self-driving-cars}{http://eijournal.com/resources/lidar-solutions-showcase/lidar-boosts-brain-power-for-self-driving-cars}


\bibitem[\protect\citename{C. Squatriglia} 2012]{Squatriglia:10}
Chuck Squatriglia
\newblock 2010-11-19.
\newblock {\em Audi's Robotic Car Climbs Pikes Peak}.
\newblock Wired.
\newblock
\href{http://www.wired.com/2010/11/audis-robotic-car-climbs-pikes-peak/}{http://www.wired.com/2010/11/audis-robotic-car-climbs-pikes-peak/}

\bibitem[\protect\citename{P. Beeson} 2009]{Beeson:09}
Patrick Beeson
\newblock 2009.
\newblock {\em Autonomous Vehicles - Lesson 2: Location Sensing}.
\newblock University of Texas.
\newblock
\href{https://www.cs.utexas.edu/~pstone/Courses/393Rfall09/resources/sensing.pdf}{https://www.cs.utexas.edu/{\textasciitilde}pstone/Courses/393Rfall09/resources/sensing.pdf}

\bibitem[\protect\citename{A. Davies} 2015]{Davies:15}
Alex Davies
\newblock 2015-04-22.
\newblock {\em Turns Out the Hardware in Self-Driving Cars Is Pretty Cheap}.
\newblock Wired.
\newblock
\href{http://www.wired.com/2015/04/cost-of-sensors-autonomous-cars/}{http://www.wired.com/2015/04/cost-of-sensors-autonomous-cars/}

\bibitem[\protect\citename{A. Iliaifar} 2013]{Iliaifar:13}
Amir Iliaifar
\newblock 2013-02-06.
\newblock {\em LIDAR, lasers, and logic: Anatomy of an autonomous vehicle}.
\newblock Digital Trends.
\newblock
\href{http://www.digitaltrends.com/cars/lidar-lasers-and-beefed-up-computers-the-intricate-anatomy-of-an-autonomous-vehicle/}{http://www.digitaltrends.com/cars/lidar-lasers-and-beefed-up-computers-the-intricate-anatomy-of-an-autonomous-vehicle/}

% ---------------------------------------------------------------
% The following citations are just examples and ARE TO BE REMOVED
% ---------------------------------------------------------------

\bibitem[\protect\citename{Aho and Ullman}1972]{Aho:72}
Alfred~V. Aho and Jeffrey~D. Ullman.
\newblock 1972.
\newblock {\em The Theory of Parsing, Translation and Compiling}, volume~1.
\newblock Prentice-{Hall}, Englewood Cliffs, NJ.

\bibitem[\protect\citename{{American Psychological Association}}1983]{APA:83}
{American Psychological Association}.
\newblock 1983.
\newblock {\em Publications Manual}.
\newblock American Psychological Association, Washington, DC.

\bibitem[\protect\citename{{Association for Computing Machinery}}1983]{ACM:83}
{Association for Computing Machinery}.
\newblock 1983.
\newblock {\em Computing Reviews}, 24(11):503--512.

\bibitem[\protect\citename{Chandra \bgroup et al.\egroup }1981]{Chandra:81}
Ashok~K. Chandra, Dexter~C. Kozen, and Larry~J. Stockmeyer.
\newblock 1981.
\newblock Alternation.
\newblock {\em Journal of the Association for Computing Machinery},
  28(1):114--133.

\bibitem[\protect\citename{Gusfield}1997]{Gusfield:97}
Dan Gusfield.
\newblock 1997.
\newblock {\em Algorithms on Strings, Trees and Sequences}.
\newblock Cambridge University Press, Cambridge, UK.

\end{thebibliography}

\end{document}

%%% Local Variables:
%%% mode: latex
%%% TeX-master: t
%%% End:
